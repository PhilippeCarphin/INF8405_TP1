\documentclass[12pt]{article}

\usepackage{geometry}
\usepackage{apacite}
\usepackage{setspace}
\linespread{1.5}
\geometry{letterpaper,tmargin=1in,bmargin=1in,lmargin=1in,rmargin=1in}

\usepackage[utf8]{inputenc}
\usepackage[T1]{fontenc}
\usepackage[frenchb]{babel}

\author{a}
\title{a}
\usepackage[final]{pdfpages}

\begin{document}
\includepdf[pages=1]{page_titre.pdf}
\setcounter{page}{1}

\tableofcontents

\section{Introduction}

\section{Présentation technique}

Dans cette section nous présentons les diverses classes et modules impliqués
dans l'application.  Premièrement, nous présentons quelques classes de
librairies importantes utilisées par l'application.  Ensuite, nous présentons
les classes que nous avons écrites pour l'application elle même.

\subsection{Classes de librairie}

Les classes importantes utilisées sont la classe wifiManager et ScanResult pour
la recherche des points d'acces WiFi à proximité.  Pour la carte et l'API Google
Maps, nous utilisons la classe GoogleMap et les classes GoogleMap.Marker et
GoogleMap.MarkerOptions.

\subsubsection{Classes utilisées pour le Wifi}

WifiManager... attributs et méthodes importantes

ScanResult ... attributs et méthodes importantes

\subsubsection{Classes utilisées pour les Maps}

GoogleMap attributs et méthodes importantes

Marker MarkerOptions ... attributs et méthodes importantes

\subsection{Classes propres à l'application}

Dans cette section nous présentons les classes créées pour notre application
ainsi que leurs méthodes.  Nous expliquerons aussi les choix de conception
pertinents pour comprendre ces classes.

\section{Difficultés rencontrées}

Les trois auteurs n'avaient pas d'expérience préalable avec le dévelopement
Android.  Nous avons donc eu une bonne mesure de difficultés.  Cette section les
énumère en détail.



\section{Critiques et suggestions}
\section{Conclusion}
Example citation: \citeA{google}.

\bibliographystyle{apacite}
\bibliography{bibdb}
\end{document}
