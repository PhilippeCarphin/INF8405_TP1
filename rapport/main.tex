\documentclass[12pt]{article}

\usepackage{geometry}
\usepackage{apacite}
\usepackage{setspace}
\linespread{1.5}
\geometry{letterpaper,tmargin=1in,bmargin=1in,lmargin=1in,rmargin=1in}

\usepackage[utf8]{inputenc}
\usepackage[T1]{fontenc}
\usepackage[frenchb]{babel}
\usepackage{float}

\author{a}
\title{a}
\usepackage[final]{pdfpages}
\newcommand\code[1]{\texttt{#1}}
\newcommand{\interfacefigheight}{2in}

\begin{document}
\includepdf[pages=1]{page_titre.pdf}
\setcounter{page}{1}

\tableofcontents

\section{Introduction}

	Dans le premier travail pratique d'informatique mobile, nous avons du
	apprendre à trouver des points d'accès WiFi et le placer sur une carte
	Google Map.

	Pour ce faire, nous avons réalisé une application Android qui présente une
	liste des points d'accès disponibles ainsi que leurs informations et offre
	quelques autres fonctionnalités comme le partage et l'ajout de favoris.

	Le rapport présente trois parties.  La présentation technique montre
	l'interface de l'application et décrit son fonctionnement interne.  Les
	sections suivantes présentent des difficultés rencontrées et des critiques
	et suggestions.

\section{Présentation technique}

	Dans cette section nous présentons, d'une part, l'interface de notre
	application.  Ensuite, nous présentons quelques classes de librairies
	importantes utilisées par l'application.  Finalement, nous présentons
	les classes que nous avons écrites pour l'application elle même.

\subsection{Interface utilisateur}

	L'écran d'accueil présente trois zones.  La zone de carte géographique, la
	barre du haut et la barre de droite (voir figure \ref{fig_start_screen}).
	Initialement, la barre de droite présente une liste des réseaux
	disponibles.  Cette liste peut être mise à jour en cliquant sur le
	bouton scan.

	\begin{figure}[h]
		\centering
		\includegraphics[height=\interfacefigheight]{./interface/start_screen.png}
		\caption{\parbox[t]{3in}{Écran d'accueil de l'application}}
		\label{fig_start_screen}
	\end{figure}

	Lorsqu'on clique soit sur un marqueur de la carte ou su un élément de la
	liste de la barre de droite, les informations du point d'accès sélectionné
	sont affichées dans la barre de droite (voir figure \ref{fig_select_marker}).

	\begin{figure}[h]
		\centering
		\includegraphics[height=\interfacefigheight]{./interface/select_marker.png}
		\caption{\parbox[t]{3in}{Cliquer sur un marqueur ou élément de la liste
		fait apparaître les détails de celui-ci}}
		\label{fig_select_marker}
	\end{figure}

	La barre de droite contient aussi des boutons pour ajouter le point d'accès
	aux favoris ou le partager.  Le bouton partager permettra de partager sur
	Facebook, Gmail ou un autre service installé sur notre appareil.

	Le bouton Ajouter Aux Favoris permettra d'ajouter le point d'accès à une
	liste de favoris.  Cette liste est accessible en cliquant sur le bouton
	Favoris de la barre du haut.

	\begin{figure}[h]
		\centering
		\includegraphics[height=\interfacefigheight]{./interface/show_favorites.png}
		\caption{\parbox[t]{3in}{Cliquer sur le bouton Favoris montre la liste
		des favoris}}
		\label{fig_favorites}
	\end{figure}

\subsection{Classes de librairie}

	Les classes importantes utilisées sont la classe wifiManager et ScanResult
	pour la recherche des points d'acces WiFi à proximité.  Pour la carte et
	l'API Google Maps, nous utilisons la classe GoogleMap et les classes
	GoogleMap.Marker et GoogleMap.MarkerOptions.

\subsubsection{Classes utilisées pour le Wifi}

	La recherche de points d'accès est faite à l'aide de la classe
	\code{WifiManager}.  Celle-ci demande les points d'accès disponible au
	dispositif avec la fonction \code{getScanResults()} qui retourne une liste
	d'objet \code{ScanResult}.

	Ceci requiert une certaine mise-en-place, notament la création d'un
	\code{WifiScanReceiver} et l'enregistrement de celui-ci avec la fonction
	\code{registerReceiver()} de la classe \code{Activity}.

\subsubsection{Classes utilisées pour les Maps}

	La classe \code{GoogleMap} est notre point d'accès aux APIs de carte
	géographique de Google.  On pose un fragment dans le \code{layout} de notre
	activté.  Ceci demande que notre classe implémente l'interface
	\code{onMapReadyCallback} qui demande que la fonction ait une méthode
	\code{onMapReady()}.


\subsection{Classes propres à l'application}

	Dans cette section nous présentons les classes créées pour notre application
	ainsi que leurs méthodes.  Nous expliquerons aussi les choix de conception
	pertinents pour comprendre ces classes.

	Notre application fonctionne avec NN classes principales.  La plus
	importante est la classe \code{MapsActivity} qui est la classe maitresse de
	l'application.

	Parmis ses attributs, les suivants sont les plus importants:
	\begin{description}
		\item[\code{mMap}]

			Un objet de type \code{com.google.android.gms.maps.GoogleMap} qui
			nous offre la communication avec les service de Google et nous donne
			un carte géographique avec laquelle nous interagissons.

		\item[\code{PointsAcces}]

			Une liste de points d'accès trouvés par \code{WifiManager}.  La
			classe \code{PointAcces} est un adaptateur de la classe
			\code{ScanResult}.

		\item[\code{FragmentListePointsAcces}]

			Un fragment qui hérite de \code{ListFragment} et qui s'occupe
			d'afficher les éléments de notre liste de \code{PointAcces}.

		\item[\code{FragmentDetailsPointsAcces}]

			Ce fragment afficher les détails d'un \code{PointAcces} dans une
			vue et des boutons dans une autre vue.  Un fragment de ce type est
			créé lorsqu'un point d'accès est sélectionné.
	\end{description}

	Ces classes coopèrent ensemble pour changer lorsqu'un point d'accès est
	sélectionné.  La figure \ref{fig_interaction} résume les interactions qui
	arrivent lorsqu'un point d'accès est sélectionné soit en cliquant sur un
	marqueur dans la carte ou en cliquant sur un élément de la liste.

	On peut sélectionner un point d'accès en cliquant dans la liste ou en
	cliquant sur son marqueur dans la carte.  Cette action cause un appel à la
	méthode \code{onPointAccesSelected()} de \code{MapsActivity}.  Cette
	fonction remplace le fragment de droite par un fragment contenant les
	détails du point d'accès ainsi que quelques boutons pour permettre des
	actons.

	\begin{figure}[H]
		\centering
		\quad\includegraphics[height=2.5in]{communication_diag.pdf}
		\parbox[t]{4in}{\caption{Diagramme des interactions de sélection de point d'accès}\label{fig_interaction}}
	\end{figure}

	% Lorsqu'un marqueur est cliqué, la fonction de rappel \code{onMarkerClick()}
	% est appelée et cette fonction appelle \code{onPointAccesSelected}.  Pour la
	% liste, elle appelle directement l


\section{Difficultés rencontrées}

	Les trois auteurs n'avaient pas d'expérience préalable avec le dévelopement
	Android.  Nous avons donc eu une bonne mesure de difficultés.  Cette section
	les énumère en détail.

\subsection{Conflits de versions}

	{\center \og There's more than one way to skin a cat.\fg \\}
		\begin{flushright}
			{\flushright Mark Twain (Connecticut Yankee in King Arthur’s Court)
			1889}
		\end{flushright}

	L'écosystème Android est en constante évolution et la compatibilité vers
	l'arrière n'est pas toujours maintenue.  Certaines classes de support nous
	permettent d'utiliser d'anciennes classes lorsqu'on en a besoin.

	Par contre, étant donné notre inexpérience avec Android, nous avons du
	recourir à des tutoriels en-ligne.  Il est arrivé souvent que solution
	donnée par un tutoriel ne fonctionne qu'avec les fragments de type
	\code{android.app.Fragment} alors on change le \code{import} au début du
	fichier.

	Mais hélas, ceci cause des erreurs dans des bouts de code déjà écrits qui,
	sans qu'on le sache, utilisent des fragments de type
	\code{android.support.v4.app.Fragment}.

	Heureusement, l'IDE Android Studio est très bon pour nous montrer ces
	erreurs et même souvent, il peut nous a proposé des solutions.

\subsection{Accès à la BD de Google}

	Puisque nous n'avons pas accès à la base de donnée de Google qui fait
	correspondre des adresses MAC à des coordonné géographiques, nous avons du
	distribuer les marqueurs aléatoirement autour de l'emplacement de
	l'usager.

\subsection{Méthodes de cycle de vie}

	Une des difficultés majeure de l'apprentissage d'Android est la
	compréhension des méthodes de cycle de vie des activités et fragments.

\section{Critiques et suggestions}

	Voici quelques critiques et suggestions pour ce travail pratique.  D'une
	part, il serait bien inviter les étudiants à lire de la documentation de
	base avant de commencer. Ensuite, on pourrait considérer d'autres
	plateformes de développement. Et finalement, laisser plus de liberté quant à
	la façon d'implémenter l'application.

\subsection{Documentation suggérée}

	La documentation suggérée s'adresse à des gens qui sont déjà familiers avec
	le fonctionnement d'Andoid.  Nous avons fait l'erreur de croire qu'elle
	serait suffisante.  Nous avons donc commencé à coder notre application avec
	une compréhension déficiente.  Ceci a mené à une architecture mal conçue.
	Nous aurions du commencer par lire quelque guides de la page Android
	Developer Guides \citeA{androidDevGuides}.

	Par exemple, nous avons consulté les guides Android Lifecycle Methods
	\citeA{androidLifecycle} et Android Fragments Guide \citeA{androidFragments}
	mais nous avons consulté ces guides après avoir écrit beaucoup de code.

	Il aurait été bien de conseiller fortement aux étudiants impatients de
	commencer par ces guides.

\subsection{Plateforme de développement}

	Il aurait été intéressant de permettre d'utiliser d'autres plateformes comme
	React-Native ou Xamarin pour le développement de l'application.  Nous
	comprenons que la chargée de laboratoire doit être familière avec la
	plateforme utilisée pour pouvoir corriger notre TP donc il est normal que
	cette suggestion ne puisse pas être prise en compte.

\subsection{Fragments}

	Pour des néophytes d'Android, l'utilisation de fragments est plutôt
	difficile à apprendre.  Il aurait été peut-être plus facile de réaliser
	l'application demandée en utilisant plusieurs activités.

\section{Conclusion}

	Dans ce TP nous avons acquis beaucoup de connaissances sur le fonctionnement
	d'Android.  Les erreurs qui nous ont causé du trouble ne serons pas répétées
	dans la prochaine application.

	Nous avons aussi approfondis nos connaissances sur les APIs de GoogleMaps et
	WifiManager.

\begin{raggedright}
\bibliographystyle{apacite}
\bibliography{bibdb}
\end{raggedright}
\end{document}
